\documentclass{article}

%% Page Margins %%
\usepackage{geometry}
\geometry{
    top = 0.75in,
    bottom = 0.75in,
    right = 0.75in,
    left = 0.75in,
}

\usepackage{amsmath}
\usepackage{graphicx}
\usepackage{parskip}

\title{Assembly Project: Breakout}

% TODO: Enter your name
\author{Jason Zhang}

\begin{document}
\maketitle

\section{Instruction and Summary}

\begin{enumerate}

    \item Which milestones were implemented? 
    % TODO: List the milestone(s) and in the case of 
    %       Milestones 4 & 5, list what features you 
    %       implemented, sorted into easy and hard 
    %       categories.
    Milestone 1, 2, 3

    \item How to view the game:
    % TODO: specify the pixes/unit, width and height of 
    %       your game, etc.  NOTE: list these details in
    %       the header of your breakout.asm file too!
    
    \begin{enumerate}

    \item Unit width in pixels:       8
    \item Unit height in pixels:      8
    \item Display width in pixels:    512
    \item Base Address for Display:   0x10008000 (\$gp)


    \end{enumerate}

    

\begin{figure}[ht!]
    \centering
    \includegraphics[width=0.3\textwidth]{name.png}
    \caption{caption}
    \label{Instructions}
\end{figure}

\item Game Summary:
% TODO: Tell us a little about your game.
\begin{itemize}
\item press 1 to start
\item the game now support exit(press q)
\item the game would black out if the player lose
\item collusion is supported
\item Support “multiple lives” (3) so that the player can continue the game multiple times. The state of the game (i.e., broken bricks) are retained for subsequent attempts.
\item Allow the user to pause the game by pressing the keyboard key p. and press other keys to continue.
\item Add ‘unbreakable’ bricks.( two grey in middle)
\item Add a second paddle that is controlled by a second player using different keys.(use comma and dot to control)
\item Allow the player to launch the ball at the beginning of each attempt.(use W, A, S, D to control movement of the ball)
\item Require bricks be hit by the ball multiple times before breaking. 

\end{itemize}

    
\end{enumerate}

\section{Attribution Table}
% TODO: If you worked in partners, tell us who was 
%       responsible for which features. Some reweighting 
%       might be possible in cases where one group member
%       deserves extra credit for the work they put in.

\begin{center}
\begin{tabular}{|| c | c ||}
\hline
 Student 1 (Name and student number) &  Student 2 (Name and student number) \\ 
 \hline
 Task & Task\\
 \hline
 Task & Task\\
 \hline
 Task & Task\\ 
 \hline
 Task & Task\\ 
 \hline
 Task & Task\\
 \hline
 Task & Task\\  
 \hline
\end{tabular}
\end{center}

% TODO: Fill out the remainder of the document as you see 
%       fit, including as much detail as you think 
%       necessary to better understand your code. 
%       You can add extra sections and subsections to 
%       help us understand why you deserve marks for 
%       features that were more challenging than they
%       might initially seem.


\end{document}